\documentclass{article}
\usepackage[utf8]{inputenc} %useful to type directly diacritic characters

\title{Pseudo-thermal Light and Ground Glass}
\author{Berke Vow Ricketti}
\begin{document}
\maketitle{}


\bibliographystyle{ieeetr}
\bibliography{references}

\section{Abstract}

I made this and put it on GitHub.
Pseudo-thermal light sources can be generated by using a conventional, monochromatic, highly coherent laser source and a rotating ground glass plate. The temporal coherence, $\tau_{c}$, second-order correlation function, $g^{(2)}$, and higher-order correlation functions, $g^{(N)}$, can be calculated in set-ups using these ground glass plates. Calculations from research literature are compared to actual thermal light measurements, showing that $\tau_{c}$ for pseudo-thermal light can not get as low as that of thermal light, and $g^{(2)}$ for pseudo-thermal light can be tuned to values above $g^{(2)} = 2$ for thermal light. Potential avenues for future research are also discussed.

\section{Introduction}

Two characteristics of conventional thermal light sources that restrict their
usefulness in experimental set-ups are their extremely low coherence time and
their low intensities \cite{Spiller2014}. The coherence time of thermal light is
always shorter than $10^{-8}s$, with some sources begin less than $1 ps$ ($10^{-12}$)
and sunlight at approximately $1.3fs$ ($10^{-15}$) \cite{Spiller2014,Deutsch}.

Ground glass can be used to reduce the spatial coherence of a laser light source. By using a spinning ground glass plate, a monochromatic source can ``mimic the effects of a thermal source.'' \cite{Grider1996}. This also allows for the use of common photodetecors, which would otherwise not be able to work at the picosecond or femtosecond coherence time of thermal light.

The method for generating such a pseudo-thermal light source with tunable
temporal coherence was originally developed by Martienssen and Spiller \cite{Spiller2014}. Stranger et al have documented extensive methods for how to reduce the spatial coherence of a beam \cite{Ndersson2017}.

\section{Methods}

Pseudo-thermal light can be generated using an ideal laser source and a ground glass plate or a diffusive screen \cite{Ndersson2017}. Laser light follows Poissonian statistics where as thermal light follows Bose-Einstein statistics \cite{Grider1996}.

The ground glass plate or diffusive screen contain randomly distributed scattering centers that scatter incoming light, causing constructive and destructive interference on a Poissonian (laser) source and leads to a Bose-Einstein distribution.

The velocity of the ground glass plate is variable, allowing a change in the scattering centers seen by the laser, and can be used to adjust the coherence time of the output laser.

\section{Results}

For an ideal Gaussian beam profile, such as a He-Ne laser operating in a single
TEM\textsubscript{00}, the coherence time, $\tau_c$, of the pseudo-thermal light
generated using a ground glass plate can be calculated as:
\begin{equation}
  \tau_{c} = \sqrt{\pi}\frac{\omega_0}{v} = \frac{\omega_0}{2\sqrt{\pi}r\nu}
\label{eq1}\end{equation}
where $\omega_0$ is the radius of the beam at the ground glass plate, $v$ is the tangential
speed of the disk at this point, $r$ is the distance from the rotational axis of the ground glass plate
to the laser point, and $\nu$ is rotation frequency of the ground glass plate \cite{Lutz}.

Equivalently, the coherence time may also be calculated when considering the
focal length, $f$, used to focus the beam. An Nd:YAG laser is also used as an acceptable
laser source by Maroti in this derivation.
\begin{equation}
  \tau_{c} = \frac{\omega_{0}}{v} \cdot \frac{4\pi}{\sqrt{2\cdot ln 2 (1 + \frac{4 k^2 \omega_0^4}{f^2})}}
\label{eq2}\end{equation}
where $k$ is the wave number and $f$ is the focal length of the lens, \cite{Maroti2013}.

This allows for the creation of a monochromatic pseudo-thermal light source with
a higher intensity (much closer to a laser, rather than a conventional thermal source)
and a variable coherence time that can be tuned by changing the rotational velocity of the ground glass plate.

\subsection{Pseudo-thermal Light vs. Thermal Light and Sunlight}

Experimentally, it has been shown that this methodology yields coherence times between 1 sec
and $10^{-5}s$ \cite{Spiller2014}. This is many orders of magnitude away from both thermal light
($10^{-8}s$) and sunlight ($\approx 10^{-15}s$) \cite{Kano1962}.

A quick calculation using Eq. \ref{eq1} shows that such a methodology would be impossible
for achieving a coherence time comparable to that of sunlight at the femtosecond scale.

Assume a coherence time of $\tau_c = 10^{-15}s$ and a beam radius of roughly 250
nm for a laser with wavelength of around 500 nm. Thus we can calculate the tangential velocity
of the ground glass plate as:
\begin{equation}
  v = \sqrt{\pi}\frac{\omega_0}{\tau_c} = \sqrt{\pi}\frac{250 \cdot 10^{-9}}{10^{-15}} \simeq 4\cdot10^{8} m/s
\end{equation}
or faster than the speed of light.

\subsection{Degree of Coherence of Pseudo-thermal Light}

Research has been done on the degree of coherence for pseudo-thermal light.

Bai et al. used a rotating ground glass plate, two lasers, and polarizers
to recreate Hanbury Brown-Twiss effects, and thus, $g^{(2)}$ measurements. By using rotating ground glass plates,
they generate beams of pseudo-thermal light \cite{Bai2017}.

It has also been shown that the $g^{(2)}$ of pseudo-thermal light could also be tuned through the use of multiple rotating ground glass plates \cite{Bai,Zhou2017}. These were shown in two-photon superbunching experiments to increase $g^{(2)} > 2$ above the second order coherence of thermal light. The scheme can be expanded to three- and multi-photon superbunching to study $g^{(3)}$ and $g^{(N)}$.

While most experiments use the Hanbury Brown-Twiss effect to measure the second order correlation function, it has also been shown that a Hong-Ou-Mandel interferometer can be used to plot the $g^{(2)}$ function \cite{Liu2013}.

\section{Possible Avenues of Future Research}

Experiments that use rotating ground glass plates to generate pseudo-thermal light generally use He-Ne or Nd:YAG lasers as their high intensities make them ideal when dealing with common photodetectors. It has been noted that, if a decrease in intensity is accepted, the generation of pseudo-thermal light can be achieved using a ``standard arc lamp and using an interference filter'' \cite{Spiller2014}.

However, at the time of writing, no research on the uses of supercontinuum laser sources with rotating ground glass plates were found. If the purpose of the rotating ground glass plate is to act as scattering centers for the laser/light source, it would be interesting to explore what would happen with a broadband/supercontinuum light source.

\section{Conclusions}

Pseudo-thermal light is a useful alternative to natural thermal light in many experimental applications due to the low intensity of natural thermal light and the inability of common photodetectors to be able to work with such short coherence times. Pseudo-thermal light generation involves using a monochromatic laser source focused through a rotating ground glass plate to turn the original Poissonian statistical source into a Bose-Einsteinian statistical source. The coherence time of the pseudo-thermal light is tunable and linked to the rotational speed of the ground glass plate. The second-, third-, and higher-order correlation functions can be calculated using pseudo-thermal light. Psuedo-thermal light generation is not limited to He-Ne or Nd:YAG lasers, but can also be achieved with arc lamps. However, no literature was found using supercontinuum or broadband sources, and this may be an interesting research topic in the future.
\end{document}
